\subsection{TaskerCLI}
  
  
  \includegraphics{image1.gif}

  We are using the Data Access Object design pattern, to abstract our database contents and more easily access it from within the Java client.
  We've abstracted our database tables into Java entities, as an example, in the image above TaskElement.java corresponds to the TaskElement table in both databases.
  
  
  
  \includegraphics{image2.gif}
  
  In order to access the abstracted data from the database we use a selection of data access classes:
  These classes apply the DAO pattern, and call methods in order to send properly formatted SQL queries to the appropriate database, and present the result of the queries in the form of the previously described entities. 
  This object is then used by the UI classes in order to generate the view.

  \subsubsection{Description of each significant Class}

  \begin{itemize}
  		\item \textbf{TeamMember} class abstracts the TeamMember table in both databases. Its instance variables correspond to columns inside 
  		databases and can be accessed and modified using the getters and setters provided. Instance of this class will be used to populate
  		the user interface with data in the program.
  		\item \textbf{Task} class abstracts the Task table in both databases. Its instance variables correspond to collumns inside 
  		databases and can be accessed and modified using the getters and setters provided. Each task has an associated list of task elements. The status of a task is stored as an integer in the database and is then transformed to an appropriate \textbf{TaskStatus} Enum inside the program.
  		\item \textbf{TaskElement} class abstracts the TaskElement table in both databases. Its instance variables correspond to collumns inside 
  		databases and can be accessed and modified using the getters and setters provided. 
  		\item \textbf{TeamMemberDB} class is one of the data access classes in the DAO pattern. It provides useful methods which can be used to retrieve and update the \textbf{TeamMember} objects in both databases. For example the \texttt{selectTeamMemberByEmail} method will as expected acquire the connection to either the MySQL or SQLite database, prepare a suitable query statement, construct the \textbf{TeamMember}
  		object and return it back to the caller. This is a really powerful concept which abstracts the raw SQL from the graphical user interface. After we make some changes to the object, we can use a single method \texttt{updateTeamMember} to put it back to the database. 
  		\item \textbf{TaskDB} class is very similiar to the previously described \textbf{TeamMemberDB} class. It also is a part of the DAO pattern and provides direct access to the information stored in the task table in both databases. It will be typically only used by the previosuly mentioned \textbf{TeamMemberDB} to construct the complete \textbf{TeamMember} object.
  		\item \textbf{TaskElementDB} class is basically the same as the previous one, but it enables the access to the TaskElement table in both databases.
  		\item \textbf{ConnectionManager} class can be used to actually connect to both databases. It enables the caller to acquire a \texttt{Connection} object to either the remote MySQL (TaskerSRV), or locally run SQLite. \textbf{ConnectionManager} is capable of registering appropriate JDBC drivers depending on the connection required. 
        \item \textbf{Syncer} class contains the logic for conflict resolution between the contents of the two databases within its \texttt{sync} method. The sync method takes two \textbf{TaskMember} objects, generates a canonical TeamMember object, which is supplied to the databases and UI. The Syncer class also contains the \texttt{login} method, displaying a dialog for user login and returning an appropriate \textbf{TeamMember} object.
        The Class also contains a \textbf{Timer} object used to invoke syncing at regular intervals.
        \item \textbf{TaskerCLI} class is the main class that is run when the application begins. It calls the \textbf{Syncer} classes \texttt{login} method and passes the returned \textbf{TeamMember} onto the UI Classes.
        \item \textbf{MainFrame} extends the \textbf{JFrame} class is the primary view after the login menu, and displays a \textbf{TeamMember} object's tasks in a sortable JTable, clicking on this invokes \textbf{TaskFrame}. Also displays the \textbf{SidebarPanel}.
        \item \textbf{SidebarPanel} extends the \textbf{JFrame} class is used for controls within the \textbf{SidebarPanel}, such as logging out and searching tasks.
        \item \textbf{TaskFrame} extends the \textbf{JDialog} class and is used to display details of a task when invoked by \textbf{MainFrame}. Allows editing of a \textbf{TaskElement} and also marking of a task as complete.
  \end{itemize}
  
  

  
